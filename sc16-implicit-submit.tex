\documentclass[conference]{IEEEtran}
%===========================================================================
\usepackage{color}
\usepackage{comment}
\usepackage{graphicx}
\usepackage{amsmath}
\usepackage{amsfonts}
\usepackage{hyperref}
\usepackage{cite}
\newcommand{\fix}[1]{{\bf \textcolor {red}{#1}}}
%===========================================================================
\begin{document}
\title{Solving Large Quantities of Small Matrix Problems on Cache-Coherent SIMD Architectures}

%===========================================================================
% author names and affiliations
% use a multiple column layout for up to three different
% affiliations
\author{
\IEEEauthorblockN{Bryce Lelbach, Hans Johansen, and Samuel Williams}
\IEEEauthorblockA{Computational Research Division, Lawrence Berkeley National Laboratory, Berkeley, CA 94720\\ {\it \{balelbach, hjohansen, swwilliams\}@lbl.gov}}
%\and
%\IEEEauthorblockN{Homer Simpson}
%\IEEEauthorblockA{Twentieth Century Fox\\
%Springfield, USA\\
%Email: homer@thesimpsons.com}
}

% conference papers do not typically use \thanks and this command
% is locked out in conference mode. If really needed, such as for
% the acknowledgment of grants, issue a \IEEEoverridecommandlockouts
% after \documentclass

% for over three affiliations, or if they all won't fit within the width
% of the page, use this alternative format:
% 
%\author{\IEEEauthorblockN{Michael Shell\IEEEauthorrefmark{1},
%Homer Simpson\IEEEauthorrefmark{2},
%James Kirk\IEEEauthorrefmark{3}, 
%Montgomery Scott\IEEEauthorrefmark{3} and
%Eldon Tyrell\IEEEauthorrefmark{4}}
%\IEEEauthorblockA{\IEEEauthorrefmark{1}School of Electrical and Computer Engineering\\
%Georgia Institute of Technology,
%Atlanta, Georgia 30332--0250\\ Email: see http://www.michaelshell.org/contact.html}
%\IEEEauthorblockA{\IEEEauthorrefmark{2}Twentieth Century Fox, Springfield, USA\\
%Email: homer@thesimpsons.com}
%\IEEEauthorblockA{\IEEEauthorrefmark{3}Starfleet Academy, San Francisco, California 96678-2391\\
%Telephone: (800) 555--1212, Fax: (888) 555--1212}
%\IEEEauthorblockA{\IEEEauthorrefmark{4}Tyrell Inc., 123 Replicant Street, Los Angeles, California 90210--4321}}




% use for special paper notices
%\IEEEspecialpapernotice{(Invited Paper)}




% make the title area
\maketitle

%===========================================================================
\begin{abstract}
Horizontal Explicit/Vertical Implicit (HE/VI) methods show promise as a
scalable approach to solving global climate problems on cubed sphere
geometries. When solving problems with a high horizontal-vertical aspect ratio
HE/VI methods, it is necessary to perform a large number of small vertical
implicit solves which are independent of each other. We present a performant
approach for solving large quantities of independent matrix problems on
cache-coherent SIMD architectures. 
\end{abstract}

% no keywords


% For peer review papers, you can put extra information on the cover
% page as needed:
% \ifCLASSOPTIONpeerreview
% \begin{center} \bfseries EDICS Category: 3-BBND \end{center}
% \fi
%
% For peerreview papers, this IEEEtran command inserts a page break and
% creates the second title. It will be ignored for other modes.
\IEEEpeerreviewmaketitle

%===========================================================================
\section*{Outline}
\begin{itemize}
\item Climate apps use HE-VI model 3D+1D. Also a pattern for 2D+1D and 3D+0D chemistry
\item Results demonstrate importance of ``batching'' solves, MLK doesn't do
\item Cache coherency is important: IMEX RK accum, tiling
\item Explicit part is HO stencil op, memory b/w bound (w/ ghost cells)
\item Implicit part is non-linear solver: App requires different matrix at each i,j index 
\item Results in repeated vertical sparse banded solve
\item Solve (tridiag, banded, dense) should vectorize on i (unit stride), tiled in pencils
\item Considerations for vector alignment, tiling, and memory
\item Performance results, scaling, comparison, by platform / parameter
\item Future work: communication hiding, load imbalance, IMEX
\end{itemize}
(Bryce's email comments) Looking over the outline:
\\
It's crucial that we clearly identify the optimizations that we
believe are novel/high-impact, vs the optimizations that are
well-known to the community (even if they were not well known to us).
For example, on the outline, we have listed "IMEX RK accum, tiling".
Do we want to present the accumulation optimizations which removed
unnecessary temporaries from the time integrator? Do we feel that
optimization is novel, or is that just a common-sense thing that only
affected us because of the Chombo programming model? Likewise, do we
want to present tiling as a focal point in this paper? Tiling is a
well-known technique; we certainly can't claim tiling in general as
novel work. Is some part of our tiling approach novel? (how about
parallelizing the tile loop - is that fairly novel? I know Sam does
this in HPGMG, but is this a widely used technique?)
\\
We would be well served by applying the scientific method at this juncture:
\\
\begin{itemize}
\item Question: What concrete research question(s) are we answering?
\begin{itemize}
\item Prior research indicates that HE-VI methods show promise as a
scalable approach to solving global climate problems on cubed sphere
geometries because <explanation> [cite prior studies]. How do we
implement HE-VI methods which perform well on cache-coherent SIMD
architectures?
\item When solving a problem with a high horizontal-vertical aspect ratio
(e.g. the horizontal extents are much greater than the vertical
extents) with HE-VI methods, it is necessary to perform a large number
of small vertical implicit solves (<give examples of matrix sizes>
[citation]) which are independent of each other. <explain the type of
solves in the climate dycore - e.g. non-linear but nearly banded>
[citation]. How can we efficiently solve large quantities of small
non-linear bandedish/banded/tridiagonal matrices on cache-coherent
SIMD architectures?
\end{itemize}
\item Hypothesis: What theories did we come up with that would answer our
research question(s)?
\begin{itemize}
\item Optimal memory access patterns, management of working set sizes
(e.g. staying in cache) and efficient use of vector units are
necessary to achieve good performance on cache-coherent SIMD
architectures.
\begin{itemize}
\item Optimal memory access patterns: moving through memory in unit
stride, controlling the number of streams.
\item Management of working set sizes: tiling, reducing size of
temporaries/localizing temporaries (thread-local or otherwise)
\item Efficient use of vector units: moving through memory in unit
stride, controlling memory alignment, controlling array strides,
annotation-assisted autovectorization
\item TODO: List of individual, concrete optimizations from above.
\end{itemize}
\item Mainstream linear algebra libraries (<examples> [citation]) are not
well-suited for solving large quantities of small non-linear
bandedish/banded/tridiagonal matrices on cache-coherent SIMD
architectures because <expalanation> [cite prior studies if possible].
\end{itemize}


\item Prediction: If our hypotheses are true, what results can we expect to
see? \\
$\rightarrow$ TODO: What performance characteristics should we see with and
without each of the concrete optimizations from our hypothesis above?

\item Experiment: Investigate the predictions we've made. \\
$\rightarrow$ TODO: Methodology for benchmarking the optimizations identified
above and measuring the performance characteristics we're interested
in.

\item Analysis: What were the results of our experiments?
\end{itemize}
% 

%===========================================================================
\section{Introduction}
% 
\fix{Hans to write}
Argument for uniqueness:

- Different at every point: nonlinear iter, banded / sparse, tridiagonal, result
of Schur complement

- Algorithm doesn't require pivoting (why?)

- Vectorized (in-place) banded/tri-diag MM

- Size of matrices 30 - 100

- SIMD architectures (not SIMT GPU)

- Mixed precision ops (float reciprocal vs. double divides)

- Tiled (experiment for non-tiled code blowing out cache)

\fix{Hans to write - put in science motivation}

%===========================================================================
\section{Related Work}
\fix{Hans to write - put in }
Is this unique? Have to look at batched / tiled sparse solvers, on Intel
architecture.

MKL

Links Hans sent
\begin{itemize}
\item Built-to-order BLAS \cite{Spampinato:2014}
\item Blaze \cite{BlazeSite}
\item PLASMA \cite{PLASMASite}
\item MAGMA \cite{Haidar:2015}
\end{itemize}


etc...

%===========================================================================
\section{Optimizations}
\fix{Bryce to write}
Optimizations

%===========================================================================
\section{Experimental Setup}
\fix{Bryce to write}
Edison~\cite{Edison_website}, Cori, etc...
Compilers
Problem Size

\begin{itemize}
\item 3x3 or 30x30 or 400x400
\item Vectorized vs. MKL / non-vectorized
\item Mixed precision
\item Tiled (experiment for non-tiled code blowing out cache)
\end{itemize}

%===========================================================================
\section{Results and Analysis}
\fix{Bryce to write}

%===========================================================================
\section{Conclusion}
\fix{Bryce to write}
The conclusion goes here.


%===========================================================================
\section*{Acknowledgments}
\fix{Comment this section out before submitting... Reformat before finalizing...}
\fix{Intel IPCC ack, too}

This research used resources in Lawrence Berkeley National Laboratory and the National Energy Research Scientific Computing Center, which are supported by the U.S. Department of Energy Office of Science's Advanced Scientific Computing Research program under contract number DE-AC02-05CH11231.  
This material is based upon work supported by the U.S. Department of Energy, Office of Science, Advanced Scientific Computing Research, Scientific Discovery through Advanced Computing (SciDAC) program.
%This research used resources of the National Energy Research Scientific Computing Center (NERSC), which is supported by the Office of Science of the U.S. Department of Energy under contract DE-AC02-05CH11231.
%This research used resources of the Argonne Leadership Computing Facility at Argonne National Laboratory, which is supported by the Office of Science of the U.S. Department of Energy under contract DE-AC02-06CH11357.
%This research used resources of the Oak Ridge Leadership Facility at the Oak Ridge National Laboratory, which is supported by the Office of Science of the U.S. Department of Energy under Contract No. DE-AC05-00OR22725.


%===========================================================================
\bibliographystyle{IEEEtran}
\bibliography{sc16-implicit}
%===========================================================================


\newpage
\section*{Timeline for results}
Idea is that we'll review this list at 10a Wed when we meet.
\begin{center}
\begin{tabular}{ |l|p{5cm}|p{5cm}|p{5cm}| } 
\hline
Deadline & Bryce & Hans & Sam \\
\hline
4/20 
  & List out descriptions of experiments, data, and figures needed in the .tex file (for each hardware, test, etc.)
  & Check on Bryce's access to KNL, or back-up plan
  & Determine where we should submit (SC workshop?) \\
\hline
4/27 
  & Bryce 
  & Hans 
  & Sam \\
\hline
5/04
  & Bryce 
  & Hans 
  & Sam \\
\hline
5/11
  & Bryce 
  & Hans 
  & Sam \\
\hline
5/18
  & Bryce 
  & Hans 
  & Sam \\
\hline
5/25
  & Bryce 
  & Hans 
  & Sam \\
\hline
6/01 
  & Finish generating performance runs data 
  & Hans 
  & Sam \\
\hline
6/08 
  & Generate final plots, write results up
  & Review paper 
  & Review paper \\
\hline
6/15 
  & Finish writing results up 
  & (On vacation) 
  & Review paper \\
\hline
6/22 
  & (padding, in case of delays)
  & (On vacation) 
  & Review paper \\
\hline
6/29 
  & (padding, in case of delays)
  & (On vacation) 
  & Review paper \\
\hline
7/06 
  & Submit paper to ?? 
  & Final review before submitting
  & Final review before submitting \\
\hline
\end{tabular}
\end{center}




%===========================================================================
\end{document}


