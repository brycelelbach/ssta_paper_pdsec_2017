\documentclass[conference]{IEEEtran}
%===========================================================================
\usepackage{color}
\newcommand{\fix}[1]{\textcolor{red}{#1}}
%===========================================================================
\begin{document}
\title{Some Title Goes Here}

%===========================================================================
% author names and affiliations
% use a multiple column layout for up to three different
% affiliations
\author{
\IEEEauthorblockN{Bryce Lelbach, Hans Johansen, and Samuel Williams}
\IEEEauthorblockA{Computational Research Division, Lawrence Berkeley National Laboratory, Berkeley, CA 94720\\ {\it \{balelbach, hjohansen, swwilliams\}@lbl.gov}}
%\and
%\IEEEauthorblockN{Homer Simpson}
%\IEEEauthorblockA{Twentieth Century Fox\\
%Springfield, USA\\
%Email: homer@thesimpsons.com}
}

% conference papers do not typically use \thanks and this command
% is locked out in conference mode. If really needed, such as for
% the acknowledgment of grants, issue a \IEEEoverridecommandlockouts
% after \documentclass

% for over three affiliations, or if they all won't fit within the width
% of the page, use this alternative format:
% 
%\author{\IEEEauthorblockN{Michael Shell\IEEEauthorrefmark{1},
%Homer Simpson\IEEEauthorrefmark{2},
%James Kirk\IEEEauthorrefmark{3}, 
%Montgomery Scott\IEEEauthorrefmark{3} and
%Eldon Tyrell\IEEEauthorrefmark{4}}
%\IEEEauthorblockA{\IEEEauthorrefmark{1}School of Electrical and Computer Engineering\\
%Georgia Institute of Technology,
%Atlanta, Georgia 30332--0250\\ Email: see http://www.michaelshell.org/contact.html}
%\IEEEauthorblockA{\IEEEauthorrefmark{2}Twentieth Century Fox, Springfield, USA\\
%Email: homer@thesimpsons.com}
%\IEEEauthorblockA{\IEEEauthorrefmark{3}Starfleet Academy, San Francisco, California 96678-2391\\
%Telephone: (800) 555--1212, Fax: (888) 555--1212}
%\IEEEauthorblockA{\IEEEauthorrefmark{4}Tyrell Inc., 123 Replicant Street, Los Angeles, California 90210--4321}}




% use for special paper notices
%\IEEEspecialpapernotice{(Invited Paper)}




% make the title area
\maketitle

%===========================================================================
\begin{abstract}
The abstract goes here. We should say something clever and thought provoking.
\end{abstract}

% no keywords


% For peer review papers, you can put extra information on the cover
% page as needed:
% \ifCLASSOPTIONpeerreview
% \begin{center} \bfseries EDICS Category: 3-BBND \end{center}
% \fi
%
% For peerreview papers, this IEEEtran command inserts a page break and
% creates the second title. It will be ignored for other modes.
\IEEEpeerreviewmaketitle



%===========================================================================
\section{Introduction}
% 

%===========================================================================
\section{Related Work}
MKL

Links Hans sent

etc...

%===========================================================================
\section{Optimizations}
Optimizations

%===========================================================================
\section{Experimental Setup}
Edison, Cori, etc...
Compilers
Problem Size

%===========================================================================
\section{Results and Analysis}

%===========================================================================
\section{Conclusion}
The conclusion goes here.


%===========================================================================
\section*{Acknowledgments}
\fix{Comment this section out before submitting... Reformat before finalizing...}

This research used resources in Lawrence Berkeley National Laboratory and the National Energy Research Scientific Computing Center, which are supported by the U.S. Department of Energy Office of Science's Advanced Scientific Computing Research program under contract number DE-AC02-05CH11231.  
This material is based upon work supported by the U.S. Department of Energy, Office of Science, Office of Advanced Scientific Computing Research, Scientific Discovery through Advanced Computing (SciDAC) program.
This research used resources of the National Energy Research Scientific Computing Center (NERSC), which is supported by the Office of Science of the U.S. Department of Energy under contract DE-AC02-05CH11231.
This research used resources of the Argonne Leadership Computing Facility at Argonne National Laboratory, which is supported by the Office of Science of the U.S. Department of Energy under contract DE-AC02-06CH11357.
This research used resources of the Oak Ridge Leadership Facility at the Oak Ridge National Laboratory, which is supported by the Office of Science of the U.S. Department of Energy under Contract No. DE-AC05-00OR22725.


%===========================================================================
% Bibliography




%===========================================================================
\end{document}


